%-----------------------------------------------------------------
%	TEMA
%	!TEX root = ./../main.tex
%-----------------------------------------------------------------
\section{Conclusions}
% \todo[inline]{Write the whole thing}
% 	\begin{itemize}
% 	\item A summary of the main points (being careful not to repeat exactly what you have written before)
% 	\item Concluding statements
% 	\item Recommendations
% 	\item Predictions
% 	\item Solutions
% \end{itemize}

Even though a big part of this text is a statistical analysis that, as yet, no other author has addressed, reproducibility has been the main focus of this thesis. This in and of itself has presented a reasonable amount of challenges; reproducing an advanced scientific study can be as hard as coming up with a state-of-the-art study on your own.
That being said, the whole process has been truly satisfying as a scientist.

The results we have found in~\cref{sec:pdi-vs-sst} are a corroboration of what most climate scientists agree on: global warming is a reality with consequences that affect us all; whilst the results found in~\cref{sec:pdi-corrs} give an interesting insight on the statistical properties of tropical-cyclones and could well be a starting point of a comprehensive study using multivariate statistics.

The main limitation of this research has probably been the amount of basins analysed. Even though our results are statistically robust as such, it would be quite precipitate to generalise these rather localised results without performing an statistical analysis for each of the omitted basins in this text.
