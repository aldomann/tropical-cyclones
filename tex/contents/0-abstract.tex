%-----------------------------------------------------------------
%	ABSTRACT
%	!TEX root = ./../main.tex
%-----------------------------------------------------------------
\cleardoubleevenemptypage
\thispagestyle{empty}
\phantomsection
\addcontentsline{toc}{section}{Abstract}
\begin{abstract}
	% \begin{enumerate}[(a)]
	% 	\item Introduction. In one sentence, what’s the topic?
	% 	\item State the problem you tackle.
	% 	\item Summarize (in one sentence) why nobody else has adequately answered the research question yet.
	% 	\item Explain, in one sentence, how you tackled the research question.
	% 	\item In one sentence, how did you go about doing the research that follows from your big idea.
	% 	\item As a single sentence, what’s the key impact of your research?
	% \end{enumerate}
	This thesis describes the physical and statistical nature of tropical-cyclones (TC) in an environment of increasing sea surface temperature.
	The influence of global warming on the intensity of TCs is a rather controversial topic that has already been addressed in many statistical studies.
	The goal of this text is to replicate the results obtained by \citeauthor{Corral2010} in \cite{Corral2010} from scratch as a learning process, to revise them using updated hurricanes track and sea surface temperature (SST) databases, and to do some new statistical analyses regarding the influence of the SST on the intensity and duration of TCs.

	Given the nature of this study, the methodology is heavily rooted in programming using a language specialised in statistical computing, along with analytical description of the maths and physics behind the TCs and the SSTs.
	Our results clearly show the effects of global warming on TC occurrence, in accordance with the results previously found by \citeauthor{Corral2010}.
\end{abstract}
